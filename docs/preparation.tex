\documentclass[11pt]{article}

\usepackage{ucs}
\usepackage[utf8x]{inputenc}
\usepackage[T1]{fontenc}
\usepackage[ngerman]{babel}

\title{Prüfungsvorbereitung für die IHK-Abschlussprüfung Teil I}
\author{L.M.}

\begin{document}
\maketitle

\section{Elektrotechnik}
\subsection{Grundgrößen der Elektrotechnik}

\paragraph{Die Elektrische Spannung:}

\begin{center}
	\begin{tabular}{ ll }
		Formelzeichen & U            \\ 
		Einheit       & Volt (V)     \\
		Beispiel      & $U = 230V$ \\ 
	\end{tabular}
\end{center}

\paragraph{Die Elektrische Stromstärke:}

\begin{center}
	\begin{tabular}{ ll }
		Formelzeichen & I            \\ 
		Einheit       & Ambere (A)   \\
		Beispiel      & $I = 2A$    \\
	\end{tabular}
\end{center}

\paragraph{Der Elektrische Widerstand:}

\begin{center}
    \begin{tabular}{ ll }
		Formelzeichen & R            \\
		Einheit       & Ohm ($\Omega$)   \\
		Beispiel      & $R = 10\Omega$    \\
	\end{tabular}
\end{center}

\paragraph{Das Ohmsche Gesetz:}

\begin{center}
    \begin{equation}
        U = R \cdot I 
    \end{equation}
    \begin{equation}
        P = U \cdot I
    \end{equation}
\end{center}

\subsection{Mathematische Zusammenhänge der Reihenschaltung}

In einer \emph{Reihenschaltung} fließt der Strom durch einen Widerstand nach
dem anderen. Daraus ergeben sich folgende Zusammenhänge: 

\begin{center}
    \begin{equation}
        U_G = U_1 + U_2 + \dots + U_n 
    \end{equation}
    \begin{equation}
        I_G = I_1 = I_2 = I_3 = \dots = I_n 
    \end{equation}
    \begin{equation}
        R_G = R_1 + R_2 + \dots + R_n 
    \end{equation}
\end{center}

Da alle Elektronen, die durch Widerstand No. 1 fließen, auch durch alle
folgenden Widerstände fließen müssen. An der \emph{Stromstärke} an sich ändert 
sich nichts. Der Gesamtwiderstand und die Gesamtspannung ist aus selbigem Grund 
die \emph{Summe aller Teilwiderstände bzw. -Spannungen}.  

\subsection{Mathematische Zusammenhänge der Parallelschaltung}

In einer \emph{Parallelschaltung} werden Widerstände und/oder Kondensatoren 
parallel in einem Schaltkreis angeordnet. Bei jeder Abzweigung \emph{teilt 
sich der Strom auf}. Daraus ergeben sich folgende Zusammenhänge:

\begin{center}
    \begin{equation}
        I_G = I_1 + I_2 + \dots + I_n
    \end{equation}
    \begin{equation}
        U_G = U_1 = U_2 = \dots = U_n
    \end{equation}
    \begin{equation}
        R_G = R_1 + R_2 + \dots + R_n 
    \end{equation}
\end{center}



\section{Digitaltechnik}
\dots

\section{Netzwerktechnik}
\dots

\section{strukturierte Verkabelung}
\dots

% ITD/Wirtschafts- und Betriebslehre
\section{Das Duale Ausbildungssystem}
\dots 

\section{IT-Ausbildungsberufe beschreiben}
\dots

\section{Den Ausbildungsbetrieb beschreiben}
\dots

\section{Betriebliche Mitbestimmung (WB)}
\dots

\section{Der Tarifvertrag (WB)}
\dots 

\section{Kündigungsschutz (WB)}
\dots

\section{Das Jugendarbeitsschutzgesetz (WB)}
\dots

\section{Mutterschutz (WB)}
\dots

\section{Sozialversicherung}
\dots

\section{Einführung in die IT für Arbeitsplätze (WB)}
\dots

\section{Das Leistungsportfolio im Ausbildungsbetrieb präsentieren (WB)}
\dots

\section{Auswahlkriterien zu IT-Produkten allgemein unterscheiden}
\dots

\section{Komponenten eines Arbeitsplatzcomputers unterscheiden}
\dots

\section{Kundenanforderungen im Leistungsprozess berücksichtigen und Projektmanagement vorbereiten}
\dots

\section{Bedarfs- und Anforderungsanalysen durchführen}
\dots

\section{Pflichtenhefte erstellen / Umsetzung und Finanzierung planen}
\dots

\section{Angebote und Stundensätze kalkulieren und die Rendite berücksichtigen}
\dots 

\section{Angebotsvergleiche bei Beschaffungsmaßnahmen durchführen}
\dots

\section{Orientierung im Modellunternehmen}
\dots

\section{Serviceanfragen bearbeiten}
\dots

\section{Service Level Agreement}
\dots

\section{Mit Kunden Angemessen kommunizieren}
\dots

\section{Serviceanfragen einordnen und schnelle Hilfe anbieten (Problem Management)}
\dots

% SWD
\section{Grundlagen zur Informationssicherheit erarbeiten}
\dots

\section{Technisch-organisatorische Maßnahmen (TOM) und Beiträge zum Sicherheitskonzept erstellen}
\dots

\section{Schutzbedarfsfeststellungen anhand eines Beispielunternehmens des BSI vorbereiten}
\dots

\section{Die Binär- Dezimal- und Hexadezimalsysteme}
\dots

\section{Programmablaufpläne, Struktogramme und Pseudocode}
\dots

\section{Objektorientierte Programmierung}
\dots

\end{document}


