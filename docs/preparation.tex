\documentclass[a4paper, 12pt]{report}

\usepackage{ucs}
\usepackage[utf8x]{inputenc}
\usepackage[T1]{fontenc}
\usepackage[ngerman]{babel}
\usepackage[a4paper, left=2.5cm, right=2cm, top=2cm, bottom=2cm]{geometry}
\usepackage{amsmath}

\title{Prüfungsvorbereitung für die IHK-Abschlussprüfung Teil I}
\author{L.M.}

\begin{document}
\maketitle

\tableofcontents

\newpage
\chapter{Entwicklung vernetzter Prozesse}
\section{Elektrotechnik}
\subsection{Grundgrößen der Elektrotechnik}

\paragraph{Die Elektrische Spannung:}

\begin{center}
	\begin{tabular}{ ll }
		Formelzeichen & U            \\ 
		Einheit       & Volt (V)     \\
		Beispiel      & $U = 230V$ \\ 
	\end{tabular}
\end{center}

\paragraph{Die Elektrische Stromstärke:}

\begin{center}
	\begin{tabular}{ ll }
		Formelzeichen & I            \\ 
		Einheit       & Ambere (A)   \\
		Beispiel      & $I = 2A$    \\
	\end{tabular}
\end{center}

\paragraph{Der Elektrische Widerstand:}

\begin{center}
    \begin{tabular}{ ll }
		Formelzeichen & R            \\
		Einheit       & Ohm ($\Omega$)   \\
		Beispiel      & $R = 10\Omega$    \\
	\end{tabular}
\end{center}

\paragraph{Das Ohmsche Gesetz:}

\begin{center}
    \begin{equation}
        U = R \cdot I 
    \end{equation}
    \begin{equation}
        P = U \cdot I
    \end{equation}
\end{center}

\subsection{Mathematische Zusammenhänge der Reihenschaltung}

In einer \emph{Reihenschaltung} fließt der Strom durch einen Widerstand nach
dem anderen. Daraus ergeben sich folgende Zusammenhänge: 

\begin{center}
    \begin{equation}
        U_G = U_1 + U_2 + \dots + U_n 
    \end{equation}
    \begin{equation}
        I_G = I_1 = I_2 = I_3 = \dots = I_n 
    \end{equation}
    \begin{equation}
        R_G = R_1 + R_2 + \dots + R_n 
    \end{equation}
\end{center}

Da alle Elektronen, die durch Widerstand No. 1 fließen, auch durch alle
folgenden Widerstände fließen müssen. An der \emph{Stromstärke} an sich ändert 
sich nichts. Der Gesamtwiderstand und die Gesamtspannung ist aus selbigem Grund 
die \emph{Summe aller Teilwiderstände bzw. -Spannungen}.  

\subsection{Mathematische Zusammenhänge der Parallelschaltung}

In einer \emph{Parallelschaltung} werden Widerstände und/oder Kondensatoren 
parallel in einem Schaltkreis angeordnet. Bei jeder Abzweigung \emph{teilt 
sich der Strom auf}. Daraus ergeben sich folgende Zusammenhänge:

\begin{center}
    \begin{equation}
        I_G = I_1 + I_2 + \dots + I_n
    \end{equation}
    \begin{equation}
        U_G = U_1 = U_2 = \dots = U_n
    \end{equation}
    \begin{equation}
        \frac{1}{R_G} = \frac{1}{R_1} + \frac{1}{R_2} + \dots + \frac{1}{R_n} 
    \end{equation}
\end{center}

In der Parallelschaltung bleibt also die \emph{Spannung konstant}, und die
Stromstärke lässt sich durch die Aufteilung der Gesamtstromstärke 
\emph{aufaddieren}. Es gilt zu beachten, dass der Widerstand nicht \emph{größer}, 
sondern \emph{kleiner} wird, je mehr Widerstände parallel geschaltet werden. 
Der Grund dafür wird klar, wenn man sich den Widerstand als Tür vorstellt - es
können \emph{mehr Menschen} hindurchgehen, je mehr \emph{Türen} es gibt. 

Auch hier gilt das Ohmsche Gesetz. Für den Spezialfall von zwei parallel
geschalteten Widerständen kann die Formel für den Gesamtwiderstand umgestellt 
werden. Es gilt damit: 
\begin{center}
    \begin{equation}
        R_G = \frac{R_1 \cdot R_2}{R_1 + R_2}
    \end{equation}
\end{center}

\subsection{Elektrizitätszähler}

Elektrizitätszähler, bzw. Energiezähler oder Kilowattstundenzähler dienen zur 
Messung der elektrischen Energie, die man im Haushalt oder in der Industrie
aus dem Stromnetz bezieht. 

Die elektrische Arbeit wird direkt mit einem Elektrizitätszähler gemessen. 
Die Anzeige erfolgt digital in der Einheit kWh. 

Für jeden Zähler wird eine Zählerkonstante $C_Z$ angegeben, aus der man
entnehmen kann, wieviele Umdrehungen der Zählerscheibe (im falle eines 
mechanischen Zählers) bzw. wieviele Impulse der LED (Im Falle eines 
elektronischen Zählers) einer kWh entscpricht. 

Für den mechanischen Zähler gilt: 

\begin{center}
    \begin{equation}
        C_Z = \frac{Umdrehungen}{kWh}
    \end{equation}
\end{center}

Für den elektronischen Zähler gilt: 
\begin{center}
    \begin{equation}
        C_Z = \frac{Impulse}{kWh}
    \end{equation}
\end{center}

\newpage
Außerdem gilt folgendes über die elektrische Arbeit (Watt):
\begin{center}
    \begin{equation}
        W = \frac{n}{C_Z}
    \end{equation}
\end{center}

Wobei $n$ der Anzahl Umdrehungen bzw. Impulsen entspricht. 

Zur Messung der elektrischen Leistung zählt man die Umdrehungen/Impulse pro 
Minute und rechnet dann hoch auf die Stunde. Die Leistung errechnet sich dann
aus: 

\begin{center}
    \begin{equation}
        P = \frac{W}{t} = \frac{n}{C_Z \cdot t}
    \end{equation}
\end{center}

Oder zur direkten Berechnung in kW:

\begin{center}
    \begin{equation}
        P = \frac{1}{\frac{1}{kWh} \cdot h} = kW
    \end{equation}
\end{center}

\section{Digitaltechnik}

\subsection{Grundverknüpfungen in der Digitaltechnik}

\subsubsection{UND/AND/Konjunktion}

Das \emph{UND} ist eine Grundverknüpfung, die nach dem Prinzip 
\emph{"Wenn A und B, dann..."} funktioniert. Der Ausgang $Q$ ist dann immer 1, 
wenn die Eingänge $A$ und $B$ gleich 1 sind. Das \emph{UND} wird auch als 
\emph{Konjunktion} bezeichnet und wird im Englischen \emph{AND} genannt. 

\begin{center}
    \begin{equation}
        Q = A \land B
    \end{equation}
    \begin{tabular}{ | c | c || c | }
    	\hline
    	A & B & Q \\ \hline 
    	0 & 0 & 0 \\ \hline
    	0 & 1 & 0 \\ \hline
    	1 & 0 & 0 \\ \hline
    	1 & 1 & 1 \\ 
    	\hline
    \end{tabular}
\end{center}

\subsubsection{ODER/OR/Disjunktion}

Das \emph{ODER} ist eine Grundverknüpfung, die nach dem Prinzip 
\emph{"Wenn eine der möglichen Bedingungen war sind, dann ist das Ergebnis 
wahr"} arbeitet. Der Ausgang Q ist dann immer 1, wenn mindestens einer der 
Eingänge $A$ und $B$ 1 sind. Das \emph{ODER} wird als Disjunktion bezeichnet 
und wird im Englischen \emph{OR} genannt. 

\begin{center}
    \begin{equation}
        Q = A \lor B
    \end{equation}
   \begin{tabular}{ | c | c || c | }
    	\hline
    	A & B & Q \\ \hline
    	0 & 0 & 0 \\ \hline
    	0 & 1 & 1 \\ \hline
    	1 & 0 & 1 \\ \hline
    	1 & 1 & 1 \\ 
    	\hline
    \end{tabular}
\end{center}

\subsubsection{NICHT/NOT/Negation}

Das \emph{NICHT} ist eine Grundverknüfung, die nach dem Prinzip \emph{"Wenn ein
Zustand oder eine Aussage wahr ist, dann ist das Ergebnis unwahr"} arbeitet. 
Das \emph{NICHT} wird als Negation bezeichnet und wird im Englischen \emph{NOT}
genannt. 

\begin{center}
    \begin{equation}
        Q = \neg A
    \end{equation}
	\begin{tabular}{ | c || c | }
		\hline
		A & Q \\ \hline
		0 & 1 \\ \hline
		1 & 0 \\
		\hline
	\end{tabular}
\end{center}

\subsubsection{NICHT-UND/NAND/NUND}

Das \emph{NICHT-UND} ist eine aus \emph{UND} und \emph{NICHT} 
zusammengeschaltete Verknüpfung, welche das Ergebnis einer \emph{UND} 
Verknüpfung negiert. Das \emph{NICHT-UND} wird auch als \emph{NUND} 
bezeichnet und wird im Englischen \emph{NAND} genannt. 

\begin{center}
    \begin{equation}
        Q = \neg(A \land B)
    \end{equation}
    \begin{equation}
        \neg Q = A \land B
    \end{equation}
   \begin{tabular}{ | c | c || c | }
    	\hline
    	A & B & Q \\ \hline
    	0 & 0 & 1 \\ \hline
    	0 & 1 & 1 \\ \hline
    	1 & 0 & 1 \\ \hline
    	1 & 1 & 0 \\ 
    	\hline
    \end{tabular}
\end{center}

\subsubsection{NICHT-ODER/NOR/NODER}

Das \emph{NICHT-ODER} ist eine aus \emph{ODER} und \emph{NICHT} 
zusammengeschaltete Verknüpfung, welche das Ergebnis einer \emph{ODER} 
Verknüpfung negiert. Das \emph{NICHT-ODER} wird auch als \emph{NODER} 
bezeichnet und wird im Englischen \emph{NOR} genannt.

\begin{center}
	\begin{equation}
		Q = \neg(A \lor B)
	\end{equation}
	\begin{equation}
		\neg Q = A \lor B
	\end{equation}
   \begin{tabular}{ | c | c || c | }
    	\hline
    	A & B & Q \\ \hline
    	0 & 0 & 1 \\ \hline
    	0 & 1 & 0 \\ \hline
    	1 & 0 & 0 \\ \hline
    	1 & 1 & 0 \\ 
    	\hline
    \end{tabular}
\end{center}

\subsubsection{Exklusiv-ODER/XOR/Antivalenz}

Das \emph{Exklusiv-ODER} ist ein zusammengeschaltetes Element aus \emph{XNOR} 
und \emph{NICHT}, welches das Ergebnis einer \emph{XNOR} Verknüpfung negiert. 
Das \emph{Exklusiv-ODER} wird auch als Antivalenz bezeichnet und wird im 
Englischen \emph{XOR} genannt. 

\begin{center}
	\begin{equation}
		Q = (\neg A \land B) \lor (A \land \neg B)
	\end{equation}
   \begin{tabular}{ | c | c || c | }
    	\hline
    	A & B & Q \\ \hline
    	0 & 0 & 0 \\ \hline
    	0 & 1 & 1 \\ \hline
    	1 & 0 & 1 \\ \hline
    	1 & 1 & 0 \\ 
    	\hline
    \end{tabular}
\end{center}

\subsubsection{Exklusiv-NICHT-ODER/XNOR/Äquivalenz}

Das \emph{Exklusiv-NICHT-ODER} ist eine zusammengeschaltete Verknüpfung, dessen 
Ausgang immer dann 1 ist, wenn alle Eingänge 2 sind. Das 
\emph{Exklusiv-NICHT-ODER} wird auch als Äquivalenz bezeichnet und wird im 
Englischen \emph{XNOR} genannt. 

\begin{center}
	\begin{equation}
		Q = (A \land B) \lor (\neg A \land \neg B)
	\end{equation}
   \begin{tabular}{ | c | c || c | }
    	\hline
    	A & B & Q \\ \hline
    	0 & 0 & 1 \\ \hline
    	0 & 1 & 0 \\ \hline
    	1 & 0 & 0 \\ \hline
    	1 & 1 & 1 \\ 
    	\hline
    \end{tabular}
\end{center}

\subsection{KV-Diagramme}

\emph{KV-Diagramme} dienen dem Vereinfachen digitaler Schaltungen und 
Schaltformeln. Ein KV-Diagramm für vier Eingänge könnte wie folgt aussehen: 

\begin{center}
    \begin{tabular}{ | c | c | c | c | c | c | }
        \hline 
         & A & A & $\neg$ A & $\neg$ A &  \\ \hline
        B & & & & & D \\ \hline
        B & & & & & $\neg$ D \\ \hline
        $\neg$ B & & & & & $\neg$ D \\ \hline
        $\neg$ B & & & & & D \\ \hline
         & C & $\neg$ C & $\neg$ C & C &  \\
        \hline
    \end{tabular}
\end{center}


\newpage
\subsection{Schaltalgebra}

\subsubsection{Negation}

Für Negationen gilt: 

\begin{center}
    \begin{equation}
        Q = A \land B
    \end{equation}
    \begin{equation}
        \neg Q = \neg (A \land B)
    \end{equation}
\end{center}

Für doppelte Negationen gilt:

\begin{center}
    \begin{equation}
        Q = A \land B
    \end{equation}
    \begin{equation}
        \neg \neg Q = \neg \neg (A \land B)
    \end{equation}
\end{center}

\subsubsection{Vorrangigkeit und Bindungsstärke}

Es gelten folgende Vorrangsregelungen: 
\begin{itemize}
    \item \emph{UND} bindet stärker als \emph{ODER}
    \item Klammern binden stärker als \emph{UND}
    \item Negationszeichen binden stärker als Klammern
\end{itemize}

\subsubsection{Auflösen von Klammern}
\begin{center}
    \begin{equation}
        Q = (A \land B) \lor (C \land D) = A \land B \lor C \land D
    \end{equation}
    \begin{equation}
        Q = (A \lor B) \land (C \lor D) = A \land C \lor A \land D \lor B \land C \lor B \land D
    \end{equation}
\end{center}

\subsubsection{Gesetze nach De Morgan}

Negationszeichen, die mehrere Variablen einer Funktionsgleichung überspannen, 
kann man nur auftrennen, wenn man das Funktionszeichen nach De Morgan wechselt.

\begin{center}
    \begin{equation}
        Q = \neg (A \land B) = \neg A \lor \neg B
    \end{equation}
    \begin{equation}
        Q = \neg (A \lor B) = \neg A \land \neg B
    \end{equation}
\end{center}

Die Schaltalgebra ist auf den drei Grundverknüpfungen \emph{UND, ODER} und 
\emph{NICHT} aufgebaut. Mit diesen Grundverknüpfungen kann man beliebige 
Verknüpfungsschaltungen aufbauen. Alle anderen logischen Verknüpfungen basieren 
basieren auf einer Kombination dieser drei Grundverknüpfungen. 

Wenn man auf \emph{UND}-Verknüpfungen verzichten will, dann kann man aus
\emph{ODER}- und \emph{NICHT}-Verknüpfen beliebige Verknüpfungsschaltungen 
aufbauen. 

Wenn man auf \emph{ODER}-Verknüpfungen verzichten will, dann kann man aus 
\emph{UND}- und \emph{NICHT}-Verknüpfungen beliebige Verknüpfungsschaltungen
aufbauen.


\section{Netzwerktechnik (Hardware)}
\subsection{Strukturierte Verkabelung}

Die \emph{Strukturierte Verkabelung} fokussiert sich auf drei diskrete Bereiche:
\begin{enumerate}
    \item Tertiärbereich: \emph{innerhalb einer Etage}
    \item Sekundärbereich: \emph{zwischen den Etagen}
    \item Primärbereich: \emph{zwischen Gebäuden}
\end{enumerate}

\subsubsection{Primärbereich}
Es werden (falls nötig auch von einem Standortverteiler aus) einzelne Leitungen
zu den Gebäuden gelegt. Im \emph{Primärbereich} machen eigentlich nur
Lichtwellenleiter Sinn. \\

Zunächst besteht bei Kupferleitungen das Problem, dass diese nur maximal 100
Meter überbrücken kann. Außerdem bestehen bei Kupferleitungen, da sie den 
elektrischen Strom leiten, die Gefahr der sog. \emph{Potentialverschleppung} 
zwischen den Gebäuden. Eine galvanische Trennung wäre mit Kupfer also nicht 
realisierbar. \\

Solch eine Potentialverschleppung \emph{wird} sich höchstwahrscheinlich durch 
Brummen und Störungen auf den Telefonleitungen, Netzwerkkabeln und TV-Leitungen 
bemerkbar machen, und im schlimmsten Fall die Endgeräte schädigen. Zusätzlich 
\emph{können} in ungünstigen Fällen Menschen und Tiere durch Stromschläge 
gefährdet werden und es können Brände entstehen. \\

Mit Lichtwellenleitern wird die Datenübertragung nicht über Stromimpulse
realisiert, sondern mit Lichtimpulsen. Sie leiten keinen Strom und ermöglichen 
dadurch eine \emph{Galvanische Trennung}, wodurch eine mögliche Quelle von
elektromagnetischen Störungen vermieden werden kann. Außerdem können 
Lichtwellenleiter auf längere Strecken eingesetzt werden und haben eine 
\emph{deutlich} höhere Datenübertragungsrate. \\

Es besteht theoretisch auch die Möglichkeit, den Primärbereich via \emph{Funk}
zu realisieren, allerdings gibt es einige Nachteile: 

\begin{itemize}
    \item Langsame Datenübertragungsrate
    \item Begrenzte Reichweite
    \item Es ist eine direkte Sichtverbindung erforderlich
    \item Daten sind einfacher von Angreifern abzugreifen als bei bsp. 
        Lichtwellenleitern
\end{itemize}

\newpage
\subsubsection{Sekundärbereich}

Im \emph{Sekundärbereich} gibt es keine allgemeingültig bessere Entscheidung
zur strukturierten Verkabelung - es muss von Fall zu Fall auf Basis vielfältiger 
\emph{Faktoren} entschieden werden. Es sind sowohl Kupfer- wie auch 
Lichtwellenleiter einsetzbar und ist anhand der Gebäudegröße sowie dem zu
erwartenden Traffic zu entscheiden.

\subsubsection{Tertiärbereich}

Im \emph{Tertiärbereich} ist allgemein Kupfer zu empfehlen, da praktisch alle 
Endgeräte ein Kupferkabel mit einem \emph{RJ45}-Stecker erwarten. Die ohnehin 
relativierbaren Performanceunterschiede auf diesen kurzen Strecken werden 
durch das Wechseln auf Kupfer/RJ45-Stecker völlig negiert, da insbesondere der
RJ45-Stecker die Geschwindigkeit auf einen Gigabit pro Sekunde limitiert. 
Außerdem bieten Kupferkabel die Möglichkeit zum Betrieb mit 
\emph{Power over Ethernet (PoE)}, was mit Glasfaserleitungen unmöglich ist. 

\subsection{Lichtwellenleiter}

\emph{Lichtwellenleiter}, auch \emph{Glasfaserleitungen} genannt, sind 
Kabel aus Glasfaser, die digitale Signale üer Lichtimpulse übertragen. Sie 
bestehen u.a. aus einem \emph{Faserkern}, einem \emph{Cladding} und einem 
\emph{Coating}. 

Lichtwellenleiter kommen in verschiedenen Varianten: \emph{Singlemode} und 
\emph{Multimode}. 

\subsubsection{Physikalische Zusammenhänge}

Bevor die Einsatzgebiete von Singlemode- und Multimodeleitungen erklärt werden
können muss zuerst erklärt werden, was eine \emph{Mode} ist. \\ 

Die Lichtwellenführung basiert auf der \emph{Totalreflexion}. Dies ist ein
besonderer Fall der Lichtbrechung, die beim Wechsel von Licht von einem 
dichteren in ein weniger dichtes Medium wechselt (bsp. Brechung eines 
Lichtstrahls im Übergang von Wasser in Luft). Tritt dieser besondere Fall auf, 
dann \emph{bricht das Licht weg} vom gedachten \glqq Lot\grqq{}. Deshalb 
ist die Optische Dichte des Faserkerns auch höher, als des 
Claddings(Glasmantels). \\

Die \emph{Totalreflexion} ist jedoch nur unter \emph{diskreten 
Einkopplungswinkeln} möglich. Jeder dieser diskreten Winkel stellen einen 
getrennten Ausbreitungsweg dar - die sog. \emph{Mode}. \\

wird das Licht in verschiedenen Moden eingekoppelt, so durchlaufen verschiedene 
Lichtanteile verschiedene Strecken, wodurch sich für die verschiedenen Moden 
verschiedene Laufzeiten ergeben. Das wird auch \emph{Modendispersion} 
genannt. Ab ca. 9$\mu$m ist nur noch die Ausbreitung einer einzelnen Mode 
möglich. 

\newpage
\subsubsection{Verschiedene Arten von Lichtwellenleitern}

Ein Lichtwellenleiter, der nur eine Mode durch das Kabel lassen kann, wird 
\emph{Singlemode LWL} genannt (im Kontrast zu \emph{Multimode LWL}, wo
hunderte von Übertragungsmoden möglich sind). In einem Singlemodefiber ist somit
keine Modendispersion möglich. \\

Dadurch ist ein Singlemodefiber besser dafür geeignet, lange Strecken im 
Kilometerbereich zu überbrücken. Für kürzere Strecken sind Singlemodefiber jeoch
zu Teuer, hier reichen Multimodefiber vollkommen aus, da auf so kurzen Distanzen
die Modendispersion nur einen minimalen Impact, \emph{wenn überhaupt} hat. \\

LWL-Kabel werden überwiegend mit zwei Steckertypen eingesetzt: LC und SC. \\

SC ist ein der ältere, größere Standard, welcher weit verbreitet ist. Er wird 
mit einer GBIC-Steckdose in einem Switch, Patchpanel, etc. verwendet. \\

LC ist ein etwas neuerer Standard mit einem kleineren Formfaktor. Er wird 
mit einer SFP- bzw. einem SFP+-Steckdose genutzt. LC-Stecker gibt es in 
\emph{Simplex-} (Unidirektional) und \emph{Duplex-} (Bidirektional) Varianten. 

\subsection{WLAN}

WLAN-Funk gibt es in zwei verschiedenen Standards bzw. Frequenzbereichen: 
\emph{IEEE802.11g bzw. 2,4 GHz und IEEE802.11ac/ax bzw. 5 GHz}, wobei 5 GHz-WLAN 
der neuere ist. \\

Es gilt zu beachten, dass das 2,4 GHz-WLAN nur eine maximale Übertragunsrate von 
54 Mbit/s erlaubt. Außerdem beginnt der Frequenzbereich bei 2,4 GHz und endet 
bei 2,4835 GHz. Bei einer Frequenzbereichsbreite von $$Df = 2,4835 GHz - 2,4 
GHz = 0,0835 GHz = 83,5 MHz$$ und einer Aufteilung auf 13 Kanäle mit einem 
Freqenzbereich von jeweils 20 MHz \emph{müssen} sich die Kanäle zwangsläufig 
überlappen. Damit bleiben nur vier Kanäle, die sich nicht überlappen: 1, 5, 9 
und 13. Dieses Problem hat 5 GHz-WLAN nicht - Es stehen (zumindest in 
Deutschland) 68 Kanäle mit einer theoretischen Höchstgeschwindigkeit von 
1 Gbit/s zuur Verfügung. \\

Es sollten folgende Planungsaspekte für ein flächendeckendes WLAN beachtet 
werden: 

\begin{itemize}
    \item Optimale Anbindung durch Vernetzung aller APs über Netzwerkkabel
    \item Alle Bereiche müssen ausgeleuchtet sein
    \item Entstehende Überlappung muss durch entsprechende Kanalwahl 
        entgegengewirkt werden
    \item Name des WLAN \emph{(SSID)} muss an allen APs gleich sein
    \item Sicherheitseinstellungen \emph{(WEP/WPA(2, 3)/\dots)} sollten an allen 
        APs gleich sein
    \item Eventuell eine verringerte Abstrahlleistung wählen (bei hoher 
        Teilnehmerzahl)
    \item Bei größerer Netzteilnehmerzahl nach Möglichkeit auf einen gemeinsamen 
        Schlüssel verzichten
\end{itemize}

\subsection{Switch vs. Hub/Auswahlkriterien für einen Switch}

Ein \emph{Hub} ist ein Netzwerkgerät, welches das Netzwerk als gemeinsames 
Medium bereitstellt. Jede Nachricht wird grundsätzlich an jeden 
Netzwerkteilnehmer geschickt. Das kann für eine hohe Netzwerkauslastung und für 
Kollisionen sorgen. Außerdem ist die Sicherheit fraglich, da \emph{jede PDU} an 
\emph{jeden Netzteilnehmer} geschickt wird. \\

Ein \emph{Switch} ist ein Netzwerkgerät, welches anhand von der MAC-Adresse und 
dem eigenem ARP-Table entscheidet, für welchen Netzteilnehmer welche Nachricht 
bestimmt ist. \\

Es bestehen folgende \emph{(mögliche)} Auswahlkriterien für Switches:

\begin{itemize}
    \item Die Portanzahl (mit Blick auf die Nutzerzahl und dem zu erwartenden 
        Traffic)
    \item Portgeschwindigkeit/Bandbreite je Port/Bandbreite der internen 
        Backplane
    \item Besondere Funktionen: 
        \begin{itemize}
            \item \emph{Spanning Tree Protocol} Unterstützung
            \item VLAN
            \item LACP (Link Aggregation)
            \item Network Access Control
            \item RADIUS
        \end{itemize}
    \item Portarten \emph{(Kupfer, GBIC, SFP, SFP+)}
    \item PoE/PoE+, maximale Gesamtleistung
    \item Lüfterloses Design
    \item Energieeffizienz
    \item Managed/Unmanaged
    \item Abmessungen (z.B. 19\dq oder Desktop)
    \item Latenzzeit beim Switching
    \item Switchingarten (Store-and-Forward, Fast-Forwarding, Cut-Through)
    \item Layer-2 vs Layer-3-Switch (Letzteres bietet auch Routing-Funktionen)
    \item Stacking
\end{itemize}

% Kann mir nicht vorstellen, dass das in der IHK-Prüfung relevant wird
% \subsubsection{Switchingverfahren}

\newpage
\subsection{Netztopologien}

In der \textbf{Sterntopologie} gehen von einem Sternmittelpunkt aus sternförmig
Leitungen zu den Endpunkten. \\

Im \textbf{Erweiterten Stern} ist der Endpunkt eines Sterns widerum Mittelpunkt 
\emph{eines weiteren Sterns}. Üblich sind bei LANs drei Ebenen. Diese Topologie
ist Standard bei heutigen Verkabelungen. \\

Bei einem \textbf{Complete Mesh} ist jede Station mit jeder anderen verbunden. 
Hier besteht eine sehr hohe Redundanz, jedoch ist diese Topologie sehr 
aufwendig. \\

Bei einem \textbf{Incomplete Mesh} sind alle wichtigen Stationen mehrfach mit 
anderen Stationen verbunden. Durch die Redundanz sind diese Netze ausfallsicher.
Ein erweiterter Stern mit Querverbindungen ergibt ein solches \emph{incomplete 
mesh}. \\

Eine \textbf{Funkzelle} deckt bestimmte Bereiche  mit Funkwellen ab, z.B. 
WLAN, Bluetooth oder Mobilfunk. Die Zugriffssteuerung erfolgt über CSMA/CA. \\

Eine \textbf{Point-To-Point-/P2P-}Verbindung ist eine Verbindung zwischen zwei
Stationen. \\

Bei einem \textbf{Bus} sind alle Stationen an einer gemeinsamen Leitung 
angeschlossen. Diese Topologie ist in LANs \emph{nicht mehr üblich}. \\

Bei einem \textbf{Ring/Token-Ring} hat jede Sation eine Vorgängerstation und 
eine Nachfolgerstation. Daten werden nur \emph{in eine Richtung} verschickt. 
Der \emph{Vorteil} besteht darin, dass dies eine sichere Datenübertragung 
darstellt und dass die Wartezeit, bis gesendet werden darf, \emph{berechenbar} 
ist. Nachteilig ist die schwierigere Fehlersuche sowie der schwierigere Aufbau. 
Die Zugriffssteuerung erfolgt über \emph{Token Passing}. \\

Allgemein unterscheidet man zwischen einer \emph{physikalischen} (wie sieht die 
Verkabelung aus?) und einer \emph{logischen} Topologie (wie verläuft der 
Datenstrom?). 

\subsection{VLAN}

Es wird zwischen zwei Typen von \emph{VLANS} unterschieden. 


\subsubsection{Portbasierter VLAN (Untagged)}

Mit \emph{Portbasierten VLANs} wird ein einzelner physischer Switch auf mehrere
\emph{logische} Switche unterteilt. Für gewöhnlich werden portbasierte VLANs vor
allem in kleinen Netzwerken verwendet. Es besteht jedoch auch die Möglichkeit, 
die Installation über mehrere Switches hinweg zu realisieren. Jedes virtuelle 
Netz benötigt allerdings eine eigene Verbindung, daher müssen die Switches mit 
zwei Kabeln verbunden werden. 

\newpage
\subsubsection{Tagged VLAN}
\emph{Tagged VLANs} arbeiten im Gegensatz tu portbasierten VLANs framebasiert. 
Eine \emph{Markierung} im Frame der Nachricht, auch \emph{Tag} genannt, sorgt 
für die Zuordnung. Der \emph{Tag} enthält die Information über den 
Aufenthaltsort, signalisiert dem Switch, wo die Kommunikation stattfindet und
leitet die Nachricht entsprechend dorthin. ein Port wird dabei nicht in einem 
einzelnen, sondern mehreren VLANs zugeordnet. Eine weitere Option ist ein 
\emph{VLAN Trunk}. Dieser wird von vielen Switches angeboten und leitet VLAN
Frames für die jeweiligen VLAN IDs weiter.

\subsection{Binärprefixe}

% Tut mir leid Hr. Klimkeit, ich kann es nicht übers Herz bringen, im 21. 
% Jahrhundert noch EDV zu sagen...
Auch abseits der IT-Welt sind Präfixe (sog. \emph{SI-Präfixe}) üblich, um Zahlen
besser handhabbar zu machen, z.B. 1m anstelle von 1000mm. \\

Die dahinterstehende Mathematik beruht bekannterweise darauf, dass man durch 
den Faktor $10^3 = 1000$ dividiert bzw. multipliziert. Die üblichen Präfixe
lauten: 

\begin{itemize}
    \item Tera (Vorsatz \emph{T})
    \item Giga (Vorsatz \emph{G})
    \item Mega (Vorsatz \emph{M})
    \item Kilo (Vorsatz \emph{k})
    \item Milli (Vorsatz \emph{m})
    \item Micro (Vorsatz \emph{$\mu$})
    \item Nano (Vorsatz \emph{n})
\end{itemize}

In der o.g. Reihenfolge gilt es immer \emph{\glqq mal 1000\grqq{}} zu rechnen, 
um eine Zahl ins SI-Präfix der darunterliegenden Zeile umzurechnen, bzw. 
\emph{\glqq durch 1000\grqq{}}, um eine Zahl ins SI-Präfix der darüberliegenden 
Zeile umzurechnen, was immer eine Verschiebung des Kommas um drei Stellen nach 
rechts bzw. links entspricht. \\

Dieser Ansatz scheitert jedoch bei der Übertragung ins Binärsystem, da der 
Umrechungsfaktor \emph{$2^3 = 8$} die Zahl nicht großartig handhabbarer machen
würde. Daher sind im Binärsystem zwei unterschiedliche Ansätze üblich:

\begin{itemize}
    \item[a)] Verwendung des Umrechnungsfaktors \emph{1000}
        \begin{itemize}
            \item Vorteil: Identisches Vorgehen zur bekannten Umrechnung des 
                Dezimalsystems
            \item Nachteil: Die resultierende Zahl wird im Dezimalsystem 
                üblicherweise zu einer Kommazahl mit entsprechend erhöhtem 
                Rechenaufwand sowie einem Genauigkeitsverlust und einer 
                erschwerten Weiterverarbeitung
        \end{itemize}
    \item[b)] Verwendung eines neuen Umrechnungsfaktors der Art $2^x$
        \begin{itemize}
            \item Vorteil: Es ergeben sich im Binärsystem keine unnötigen 
                Kommazahlen
            \item Nachteil: Der Umrechnungsfaktor kann nicht mehr identisch zum
                altbekannten Umrechnungsfaktor sein
        \end{itemize}
\end{itemize}

\emph{Kurz gesagt:}  Beide Vorangehensweisen haben Vor- und Nachteile und beide 
Ansätze finden Verwendung. Bei Ansatz \emph{B} bleibt dann noch festzulegen, 
welchen x-Wert man für den Umrechnungsfaktor $2^x$ einsetzt - Das ist 
prinzipiell willkürlich, aber man hat hier sinnigerweise $x = 10$ gewählt, da 
$2^10 = 1024$ einen Umrechnungsfaktor ergibt, der sehr nahe an dem üblichen 
Umrechnungsfaktor 1000 liegt. \\

Aufgrunddessen ist es in der Informatik bzw. IT wichtig zu kennzeichnen, ob
das genutzte Präfix den \emph{1000er-Umrechnungsfaktor} oder den 
\emph{1024er-Umrechnungsfaktor} zugrunde legt. Falls der 
\emph{1024er-Umrechnungsfaktor} verwendet wurde, kommt folgende 
\emph{Binärprefixe} zum Einsatz: 

\begin{itemize}
    \item Tebi (Vorsatz \emph{Ti} statt \emph{T})
    \item Gibi (Vorsatz \emph{Gi} statt \emph{G})
    \item Mebi (Vorsatz \emph{Mi} statt \emph{M})
    \item Kibi (Vorsatz {Ki} statt \emph{k})
\end{itemize}

Wie so häufig in der IT hält sich aber leider nicht jeder an diesen 
nicht-bindenden Standard, bsp. zeigt \emph{Microsoft Windows} die 
Festplattengröße als GB an, obwohl GiB gemeint ist. Im Kontrast dazu geben
\emph{Festplattenhersteller} die Speicherkapazität in GB an \emph{und meinen 
das auch so!} \\

Aus diesem Grund muss zwischen den \emph{SI-Präfixen} und den 
\emph{Binärprefixen} umgerechnet werden können. Ein Rechenweg von TB in 
TiB sieht beispielsweise so aus: 

\begin{center}
    \begin{equation}
        \begin{split}
            2TB \cdot \frac{1000 \frac{GB}{TB}}{1024 \frac{GiB}{TiB}} 
            \cdot \frac{1000 \frac{MB}{GB}}{1024 \frac{MiB}{GiB}} 
            \cdot \frac{1000 \frac{kB}{MB}}{1024 \frac{kiB}{MiB}} 
            \cdot \frac{1000 \frac{Byte}{kB}}{1024 \frac{Byte}{kiB}} 
            = 2TB \cdot \left( \frac{1000}{1024}^4 \right) \frac{TiB}{TB} 
            \approx 1,82 TiB
        \end{split}
    \end{equation}
\end{center}



% ITD/Wirtschafts- und Betriebslehre
\chapter{Gestaltung von IT-Dienstleistungen/Wirtschafts- und Betriebslehre}

\section{Das Duale Ausbildungssystem}

\subsection{Wie entsteht ein Ausbildungsberuf?}

Die meisten Jugendlichen in Deutschland (mehr, als zwei Drittel eines 
Altersjahrgangs) beginnen nach der schule mit einer \emph{Lehre}, d.h. mit 
einer Ausbildung im \emph{Dualen System}. Dual wird dieses System genannt, weil
die Ausbildung an \emph{zwei Lernorten} stattfindet, am Arbeitsplatz im Betrieb
und in der Berufsschule. \\

Aufgrund von fortlaufenden Entwicklungen in der Berufswelt entstehen immer mehr 
neue Ausbildungsberufe, bzw. bestehende werden aktualiserte werden an die neuen 
Anforderungen angepasst. Das Verfahren für die Entwicklung neuer 
Ausbildungsberufe läuft immer in mehreren Schritten ab. Zunächst treten \emph{
Ausbildungsbetriebe, Kammern oder Gewerkschaften} an die Bundesregierung heran 
und fordern die Vermittlung von neuen Fähgkeiten, die in der Wirtschaft benötigt 
werden. Im Anschluss daran werden \emph{die bestehnden Ausbildungsberufe im 
Hinblick auf ihre Inhalte und Strukturen geprüft}. \emph{Entweder} werden dann 
Vorschläge für eine Umstrukturierung gemacht, \emph{oder} es wird ein ganz neuer 
Ausbildungsberuf eingerichtet. Die Vorschläge werden nun in einer Runde von 
\emph{Arbeitgebern, Gewerkschaften, Bund und Ländern} diskutiert. Hier werden 
die \emph{Qualifikationen} für den neuen Ausbildungsberuf festgelegt. Im 
folgenden erarbeitet das \emph{Bundesinsitut für Berufsbildung} im Auftrag des 
\emph{Bundesministeriums für Bildung und Forschung} zusammen mit den 
\emph{Arbeitgebervertretern und den Gewerkschaften} einen neuen 
\emph{Ausbildungsordnungsentwurf} für die Ausbildungsbetriebe. Die 
Sachverständigen der Länder entwickeln einen Rahmenlehrplan für die Inhalte des 
Berufsschulunterrichts. Beide Entwürfe werden miteinander abgestimmt. 
Letztendlich erlässt das Bundesministerium für Forschung und Entwicklung die 
\emph{Ausbildungsordnung} und die \emph{Kultusminister der Länder} verabschieden 
den Rahmenplan. Der Ausbildungsberuf erhält siene Gültigkeit durch die 
\emph{Staatliche Anerkennung des Bundesministers für Bildung und Forschung}. \\

Die \emph{Dualität} der Berufsausbildung basiert auf mehreren Regelungen. Für 
den \emph{Auszubildenden} wird die Ausbildung zum einen durch den 
\emph{Ausbildungsvertrag (Betriebliche Ausbildung)} und zum anderen durch die 
\emph{Berufsschulpflicht (Berufliche Schulbildung)} begründet. Die Ausbildung
wird zum einen durch die \emph{Kammern} und zum anderen durch die 
\emph{Schulaufsicht} überwacht. Finanziert wird sie durch den Ausbildungsbetrieb
und durch die Länder. 


% //TODO
% \section{IT-Ausbildungsberufe beschreiben}
% \dots

% \section{Den Ausbildungsbetrieb beschreiben}
% \dots

\section{Wichtige Inhalte in einem Kaufvertrag}

\subsection{Willenserklärungen}

Ein Kaufvertrag kommt nur dann zustande, wenn \emph{zwei inhaltlich 
Übereinstimmende Willenserlärungen} vorliegen. Der Kaufvertrag ist auf die 
\emph{Herbeiführung einer Rechtsfolge} ausgerichtet. Der Ausdruck einer 
\emph{Willenserklärung} kann ausdrücklich (z.B. bei einem schriftlichen 
Vertrag), oder auch \emph{konkludent} (schlüssiges Handeln, z.B. beim Einkauf 
bei Aldi) sein. Es gilt zu beachten, dass \emph{Schweigen i.d.R. \textbf{keine}}
Willenserklärung darstellt. Anders ist dies jedoch unter Kaufläuten. \\

\subsection{Anfragen}

Anfragen, bzw. \emph{Angebotsanfragen} dienen der Einholung von Angeboten. Diese
sind grundsätzlich \emph{\textbf{nicht} bindend}. Die Anfrage verpflichtet den 
Anfragenden also \textbf{nicht} auf das folgende Angebot des Liferanten zu 
bestellen. Es besteht außerdem \emph{keinen gesetzlich vorgeschriebenen 
Formzwang}, das heißt, dass Anfagen mündlich, schriftlich, telefonisch, per Fax, 
per E-Mail, usw. erfolgen können. Es wird zwischen \emph{Allgemeinen Anfragen} 
unterschieden. \\

\subsection{Angebot vs. Anpreisung}

Ein \emph{Angebot} ist immer an eine bestimmte Person gerichtet, während die 
\emph{Anpreisung} bsp. wie ein Preislabel bei Netto an die Allgemeinheit 
gerichtet ist. Diese sind \textbf{immer} rechtlich bindend, sofern keine 
Freizeichnungsklauseln angegeben sind \emph{(z.B. \glqq Solange der Vorrat 
reicht\grqq{}, \glqq freibleibend\grqq{}, etc)}.

\subsection{Angebotsfristen}

Sofern keine \emph{explizite} Frist angegeben ist, dann gilt folgendes: 

\begin{itemize}
    \item \emph{Bei Anwesenheit oder per Telefon} ist das Angebot nur in diesem
        Moment gültig, d.h. der Käufer muss sich \textbf{sofort} dazu 
        entscheiden, es anzunehmen bzw. abzulehnen
    \item \emph{In schriftlicher Form} kommt es darauf an. Der Verkäufer ist nur 
        so lange an das Angebot gebunden, wie er \emph{unter Verkehrsüblichen 
        Umständen} mit einer Antwort rechnen kann.
\end{itemize}

\subsection{Erlöschen der Bindung an das Angebot}

Die rechtliche Bindung an ein Angebot \emph{erlischt}, wenn:

\begin{itemize}
    \item es \emph{abgelehnt} wird, 
    \item die Bestellung \emph{vom Angebot abweicht},
    \item die Bestellung \emph{zu spät beim Anbieter ankommt} oder 
    \item der Anbieter das Angebot \emph{widerruft}, sofern der Widerruf \emph{
        vor oder spätestens zeitgleich mit dem Angebot} beim Käufer 
        eintrifft!
\end{itemize}


\section{Betriebliche Mitbestimmung (WB)}
\dots

\section{Der Tarifvertrag (WB)}
\dots 

\section{Kündigungsschutz (WB)}
\dots

\section{Das Jugendarbeitsschutzgesetz (WB)}
\dots

\section{Mutterschutz (WB)}
\dots

\section{Sozialversicherung}
\dots

\section{Einführung in die IT für Arbeitsplätze (WB)}
\dots

\section{Das Leistungsportfolio im Ausbildungsbetrieb präsentieren (WB)}
\dots

\section{Auswahlkriterien zu IT-Produkten allgemein unterscheiden}
\dots

\section{Komponenten eines Arbeitsplatzcomputers unterscheiden}
\dots

\section{Kundenanforderungen im Leistungsprozess berücksichtigen und Projektmanagement vorbereiten}
\dots

\section{Bedarfs- und Anforderungsanalysen durchführen}
\dots

\section{Pflichtenhefte erstellen / Umsetzung und Finanzierung planen}
\dots

\section{Angebote und Stundensätze kalkulieren und die Rendite berücksichtigen}
\dots 

\section{Angebotsvergleiche bei Beschaffungsmaßnahmen durchführen}
\dots

\section{Orientierung im Modellunternehmen}
\dots

\section{Serviceanfragen bearbeiten}
\dots

\section{Service Level Agreement}
\dots

\section{Mit Kunden Angemessen kommunizieren}
\dots

\section{Serviceanfragen einordnen und schnelle Hilfe anbieten (Problem Management)}
\dots


% SWD
\chapter{Softwaretechnologie und Datenmanagement}

\section{Grundlagen zur Informationssicherheit erarbeiten}
\dots

\section{Technisch-organisatorische Maßnahmen (TOM) und Beiträge zum Sicherheitskonzept erstellen}
\dots

\section{Schutzbedarfsfeststellungen anhand eines Beispielunternehmens des BSI vorbereiten}
\dots

\section{Die Binär- Dezimal- und Hexadezimalsysteme}
\dots

\section{Programmablaufpläne, Struktogramme und Pseudocode}
\dots

\section{Objektorientierte Programmierung}
\dots

\end{document}


