\documentclass[a4paper, 12pt]{article}

\usepackage{ucs}
\usepackage[utf8x]{inputenc}
\usepackage[T1]{fontenc}
\usepackage[ngerman]{babel}
\usepackage[a4paper, left=2.5cm, right=2cm, top=2cm]{geometry}

\title{Prüfungsvorbereitung für die IHK-Abschlussprüfung Teil I}
\author{L.M.}

\begin{document}
\maketitle

\section{Elektrotechnik}
\subsection{Grundgrößen der Elektrotechnik}

\paragraph{Die Elektrische Spannung:}

\begin{center}
	\begin{tabular}{ ll }
		Formelzeichen & U            \\ 
		Einheit       & Volt (V)     \\
		Beispiel      & $U = 230V$ \\ 
	\end{tabular}
\end{center}

\paragraph{Die Elektrische Stromstärke:}

\begin{center}
	\begin{tabular}{ ll }
		Formelzeichen & I            \\ 
		Einheit       & Ambere (A)   \\
		Beispiel      & $I = 2A$    \\
	\end{tabular}
\end{center}

\paragraph{Der Elektrische Widerstand:}

\begin{center}
    \begin{tabular}{ ll }
		Formelzeichen & R            \\
		Einheit       & Ohm ($\Omega$)   \\
		Beispiel      & $R = 10\Omega$    \\
	\end{tabular}
\end{center}

\paragraph{Das Ohmsche Gesetz:}

\begin{center}
    \begin{equation}
        U = R \cdot I 
    \end{equation}
    \begin{equation}
        P = U \cdot I
    \end{equation}
\end{center}

\subsection{Mathematische Zusammenhänge der Reihenschaltung}

In einer \emph{Reihenschaltung} fließt der Strom durch einen Widerstand nach
dem anderen. Daraus ergeben sich folgende Zusammenhänge: 

\begin{center}
    \begin{equation}
        U_G = U_1 + U_2 + \dots + U_n 
    \end{equation}
    \begin{equation}
        I_G = I_1 = I_2 = I_3 = \dots = I_n 
    \end{equation}
    \begin{equation}
        R_G = R_1 + R_2 + \dots + R_n 
    \end{equation}
\end{center}

Da alle Elektronen, die durch Widerstand No. 1 fließen, auch durch alle
folgenden Widerstände fließen müssen. An der \emph{Stromstärke} an sich ändert 
sich nichts. Der Gesamtwiderstand und die Gesamtspannung ist aus selbigem Grund 
die \emph{Summe aller Teilwiderstände bzw. -Spannungen}.  

\subsection{Mathematische Zusammenhänge der Parallelschaltung}

In einer \emph{Parallelschaltung} werden Widerstände und/oder Kondensatoren 
parallel in einem Schaltkreis angeordnet. Bei jeder Abzweigung \emph{teilt 
sich der Strom auf}. Daraus ergeben sich folgende Zusammenhänge:

\begin{center}
    \begin{equation}
        I_G = I_1 + I_2 + \dots + I_n
    \end{equation}
    \begin{equation}
        U_G = U_1 = U_2 = \dots = U_n
    \end{equation}
    \begin{equation}
        \frac{1}{R_G} = \frac{1}{R_1} + \frac{1}{R_2} + \dots + \frac{1}{R_n} 
    \end{equation}
\end{center}

In der Parallelschaltung bleibt also die \emph{Spannung konstant}, und die
Stromstärke lässt sich durch die Aufteilung der Gesamtstromstärke 
\emph{aufaddieren}. Es gilt zu beachten, dass der Widerstand nicht \emph{größer}, 
sondern \emph{kleiner} wird, je mehr Widerstände parallel geschaltet werden. 
Der Grund dafür wird klar, wenn man sich den Widerstand als Tür vorstellt - es
können \emph{mehr Menschen} hindurchgehen, je mehr \emph{Türen} es gibt. 

Auch hier gilt das Ohmsche Gesetz. Für den Spezialfall von zwei parallel
geschalteten Widerständen kann die Formel für den Gesamtwiderstand umgestellt 
werden. Es gilt damit: 
\begin{center}
    \begin{equation}
        R_G = \frac{R_1 \cdot R_2}{R_1 + R_2}
    \end{equation}
\end{center}

\subsection{Elektrizitätszähler}

Elektrizitätszähler, bzw. Energiezähler oder Kilowattstundenzähler dienen zur 
Messung der elektrischen Energie, die man im Haushalt oder in der Industrie
aus dem Stromnetz bezieht. 

Die elektrische Arbeit wird direkt mit einem Elektrizitätszähler gemessen. 
Die Anzeige erfolgt digital in der Einheit kWh. 

Für jeden Zähler wird eine Zählerkonstante $C_Z$ angegeben, aus der man
entnehmen kann, wieviele Umdrehungen der Zählerscheibe (im falle eines 
mechanischen Zählers) bzw. wieviele Impulse der LED (Im Falle eines 
elektronischen Zählers) einer kWh entscpricht. 

Für den mechanischen Zähler gilt: 

\begin{center}
    \begin{equation}
        C_Z = \frac{Umdrehungen}{kWh}
    \end{equation}
\end{center}

Für den elektronischen Zähler gilt: 
\begin{center}
    \begin{equation}
        C_Z = \frac{Impulse}{kWh}
    \end{equation}
\end{center}

Außerdem gilt folgendes über die elektrische Arbeit (Watt):
\begin{center}
    \begin{equation}
        W = \frac{n}{C_Z}
    \end{equation}
\end{center}

Wobei $n$ der Anzahl Umdrehungen bzw. Impulsen entspricht. 

Zur Messung der elektrischen Leistung zählt man die Umdrehungen/Impulse pro 
Minute und rechnet dann hoch auf die Stunde. Die Leistung errechnet sich dann
aus: 

\begin{center}
    \begin{equation}
        P = \frac{W}{t} = \frac{n}{C_Z \cdot t}
    \end{equation}
\end{center}

Oder zur direkten Berechnung in kW:

\begin{center}
    \begin{equation}
        P = \frac{1}{\frac{1}{kWh} \cdot h} = kW
    \end{equation}
\end{center}

\section{Digitaltechnik}

\subsection{Grundverknüpfungen in der Digitaltechnik}

\subsubsection{UND/AND/Konjunktion}

Das UND ist eine Grundverknüpfung, die nach dem Prinzip 
\emph{"Wenn A und B, dann..."} funktioniert. Der Ausgang $Q$ ist dann immer 1, 
wenn die Eingänge $A$ und $B$ gleich 1 sind. Das UND wird auch als Konjunktion
bezeichnet und wird im Englischen \emph{AND} genannt. 

\begin{center}
    \begin{equation}
        Q = A \land B
    \end{equation}
\end{center}

\subsubsection{ODER/OR/Disjunktion}

Das ODER ist eine Grundverknüpfung, die nach dem Prinzip \emph{"Wenn eine der 
möglichen Bedingungen war sind, dann ist das Ergebnis wahr"} arbeitet. Der 
Ausgang Q ist dann immer 1, wenn mindestens einer der Eingänge $A$ und $B$ 1
sind. Das ODER wird als Disjunktion bezeichnet und wird im Englischen \emph{OR}
genannt. 

\begin{center}
    \begin{equation}
        Q = A \lor B
    \end{equation}
\end{center}

\subsubsection{NICHT/NOT/Negation}

Das NICHT ist eine Grundverknüfung, die nach dem Prinzip \emph{"Wenn ein
Zustand oder eine Aussage wahr ist, dann ist das Ergebnis unwahr"} arbeitet. 
Das NICHT wird als Negation bezeichnet und wird im Englischen \emph{NOT}
genannt. 

\begin{center}
    \begin{equation}
        Q = \neg A
    \end{equation}
\end{center}

\subsubsection{NICHT-UND/NAND/NUND}


\begin{center}
    \begin{equation}
        Q = \neg(A \land B)
    \end{equation}
    \begin{equation}
        \neg Q = A \land B
    \end{equation}
\end{center}

\section{Netzwerktechnik}
\dots

\section{strukturierte Verkabelung}
\dots

% ITD/Wirtschafts- und Betriebslehre
\section{Das Duale Ausbildungssystem}
\dots 

\section{IT-Ausbildungsberufe beschreiben}
\dots

\section{Den Ausbildungsbetrieb beschreiben}
\dots

\section{Betriebliche Mitbestimmung (WB)}
\dots

\section{Der Tarifvertrag (WB)}
\dots 

\section{Kündigungsschutz (WB)}
\dots

\section{Das Jugendarbeitsschutzgesetz (WB)}
\dots

\section{Mutterschutz (WB)}
\dots

\section{Sozialversicherung}
\dots

\section{Einführung in die IT für Arbeitsplätze (WB)}
\dots

\section{Das Leistungsportfolio im Ausbildungsbetrieb präsentieren (WB)}
\dots

\section{Auswahlkriterien zu IT-Produkten allgemein unterscheiden}
\dots

\section{Komponenten eines Arbeitsplatzcomputers unterscheiden}
\dots

\section{Kundenanforderungen im Leistungsprozess berücksichtigen und Projektmanagement vorbereiten}
\dots

\section{Bedarfs- und Anforderungsanalysen durchführen}
\dots

\section{Pflichtenhefte erstellen / Umsetzung und Finanzierung planen}
\dots

\section{Angebote und Stundensätze kalkulieren und die Rendite berücksichtigen}
\dots 

\section{Angebotsvergleiche bei Beschaffungsmaßnahmen durchführen}
\dots

\section{Orientierung im Modellunternehmen}
\dots

\section{Serviceanfragen bearbeiten}
\dots

\section{Service Level Agreement}
\dots

\section{Mit Kunden Angemessen kommunizieren}
\dots

\section{Serviceanfragen einordnen und schnelle Hilfe anbieten (Problem Management)}
\dots

% SWD
\section{Grundlagen zur Informationssicherheit erarbeiten}
\dots

\section{Technisch-organisatorische Maßnahmen (TOM) und Beiträge zum Sicherheitskonzept erstellen}
\dots

\section{Schutzbedarfsfeststellungen anhand eines Beispielunternehmens des BSI vorbereiten}
\dots

\section{Die Binär- Dezimal- und Hexadezimalsysteme}
\dots

\section{Programmablaufpläne, Struktogramme und Pseudocode}
\dots

\section{Objektorientierte Programmierung}
\dots

\end{document}


